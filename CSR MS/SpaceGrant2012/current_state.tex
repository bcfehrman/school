One of the most prevalent initiatives in point cloud data algorithms is the Point Cloud Library project. The Point Cloud Library, or \emph{PCL}, is a large scale, stand alone, and open source project for both 2D and 3D point cloud processing. The PCL contains many advanced image processing functions such as 3D point cloud stitching and object recognition. Devices using the OpenNI 3D (e.g., Microsoft Kinect) interface can send their data to a program  for processing via the PCL \cite{PCL2012}.

Aldoma et al researched fast 3D feature based object recognition and pose estimation. Their aim was to take a set of CAD models that represented how we perceive an item in the world and then accurately identify each one of those objects in a real world scene using a depth sensor. The algorithm was an extension of the Viewpoint Feature Histogram (\emph{VFH}) and is more geared towards clustered environments and, as such, is dubbed the Clustered Viewpoint Feature Histogram (\emph{CVFH}). They found that with using a Kinect sensor on a set of 44 objects that their algorithm was better able to recognize objects in the presence of partial occlusion and noise than the original VFH routine \cite{Aldoma2011}

More recently, in collaboration with the PCL project, Aldoma and others have even further extended the CVFH algorithm such that they can repeatably place a reference frame on 3D objects in a scene. This reference frame is matched to a model reference frame in order to more easily determine an item's pose and to more identify the item with greater certainty. They found that a substantial improvement was made over the CVFH method alone in terms of accuracy and computational performance. They plan to add CUDA support for future iterations of their algorithms \cite{Aldoma2012}.