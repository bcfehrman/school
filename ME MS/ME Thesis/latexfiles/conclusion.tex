% A conclusion chapter

The motivation of this work was to accelerate the recovery rate of a self healing polymer by focusing acoustic energy at the recovery site. Time reversal signal processing was the algorithm chosen to achieve the localized pressure at an arbitrarily positioned crack whose location was unknown. In this work one dimensional time reversal was explored both theoretically and experimentally, with the novelty being experimental verification that time reversal is achievable in a one dimensional system such as a finite length rod. The first step was to determine if it was possible to detect when damage had occurred within the rod. The crack detection tests showed that there was large change in the wave signature of a stress wave sent from one end of the rod and recorded the other when a crack was introduced. The change in wave signature could easily be detected in software to signal the start of the time reversal process. Next, calculations were performed using models derived by Dr. Korde for both the steel and nylon rod cases \cite{Fehrman2012}. Rod segments of random lengths were used and a PZT sandwiched between the segments acted as the defect location. The theoretical models were compared with the experimental models and a decent match was seen. The difference between the theory and experiments could be due to the model not taking in to account dispersion, dissipation, PZT ringing, and other physical phenomena. Still, it was seen that the amplitude of the response at the defect location increased by using time reversal. Lastly, iterative time reversal was performed to see if the stress wave energy at the crack location would increase with successive iterations until the algorithm converged on the playback signals used in the time reversal phase. The experiments showed that the amplitude gain increased rapidly in the beginning of the time reversal process and then began to level out. The nylon rods proved to have better results than the steel rods and this was consistent with previously published work \cite{Fink1993}. 

Future work will involve modeling which includes more real life phenomena such as the transducer ringer, dispersion, dissipation, and other variables which change the shape, amplitude, phase, and other characteristics of the propagated wave. The time reversal testing should also be performed on a self healing polymer to determine if the method can in fact increase the recovery rate at a healing site.
