%Experiments

This chapter describes the experiments that were carried out to investigate the different components of one dimensional time reversal focusing at a defect location which is at an a prior unknown location. The components investigated include: i) detecting that a crack as occurred within the rod, ii) comparing theory to experiment for the interrogatory phase and for the first iteration of time reversal focusing, and iii) iterative focusing to achieve better a response at the defect until convergence is reached. All of the hardware and algorithms used are detailed and code samples are given in the appendices.

\section{Crack Detection}

The first topic investigated is if the system is able to sense when a defect has become present within the material. This step is important because it would be inefficient and potentially damaging to attempt to perform acoustic focusing when no defect is present in the system.

\subsection{Crack Detection Setup}
The setup used for the crack detection involved the use of both nylon and steel solid circular rods. Each rod was approximately $12.7 mm$ in diameter. 